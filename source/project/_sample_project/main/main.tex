\documentclass{article}

\usepackage{luatexja}

\begin{document}

\title{「科学的」方法の適用されぬ場合}
\author{中谷宇吉郎}

\maketitle

お茶のうまい出し方のような問題は、「科学的」方法を用いるとまことに簡単なことのように一寸思われる。いろいろの温度の湯をいろいろ量をかえて用い、浸出の時間をまたいろいろにかえてみて、一番うまい茶の出来る条件を一度探しておけば、後はいつでもうまい茶が出るはずである。水の問題もあるが、特別に鴨川の水を必要とする程感覚の発達した特殊の人の場合を除いては、いつも水道の水を使うということにしておけば大体良いだろうと考えておく。

実は前に家で一寸こういう簡単な実験を試みたことがある。それで大体良い条件がわかったつもりにしておいたのであるが、しばらく経って同じ茶を用いて前のような条件でやってみてもなかなかうまい茶が出ないので一寸意外に思った。これは考えてみれば何でもないことで、茶の品質が時間の経つにつれて変化することを考えなかったからである。ところが品質の方をその都度科学的にきめるとなると、結局またいろいろの条件で出してみなくてはならぬことになるので、一度条件をきめておけば良いというふうに簡単には片付かぬのである。これに似たことはいくらもあるので、生兵法の科学的方法を振り廻すのは或る場合には危険である。

しかしこの議論は本当の意味の科学的方法が悪いというのでは決してない。例えばこの場合にしても、茶の保存の方法と時間とをいろいろかえ、各種の茶について品質の変化の様子を調べておけば、問題は一応は片付くはずである。ただ、今の科学は条件をいろいろ分析して各要素についてその影響を調べてその結果を綜合して結論を出すというふうな方法を採っているので、この場合のように要素が沢山あって、目的とするところはその綜合した効果であるような時に、従来の科学的方法を適用することが利口であるかどうかという問題が残るだけである。

\end{document}
